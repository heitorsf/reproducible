A simulação computacional de modelos matemáticos tem sido utilizada em neurociências e auxiliado o estudo e a compreensão de mecanismos envolvidos no funcionamento do sistema nervoso, que podem ser difíceis por meio exclusivo da experimentação. Especificamente, para o sistema neuromuscular humano, modelos matemáticos multiescala (que representam mecanismos operando em níveis celulares e/ou moleculares até elementos macroscópicos, como os processos de geração e controle da força muscular) têm possibilitado estudos sobre os comportamentos dinâmicos de redes de motoneurônios alfa (MNs) controlando a geração de força muscular [1-5]. Estes estudos fornecem subsídios conceituais para uma maior compreensão de aspectos neurofisiológicos do controle motor, como, por exemplo: i) os papéis do recrutamento e da modulação da frequência de disparos de potenciais de ação (PAs) na relação força-EMG [1]; ii) a atuação de vias de realimentação sensorial no controle da força muscular [2]; iii) o efeito de condições patológicas no controle reflexo da força [3]; e iv) o efeito de entradas neuromodulatórias na atividade de MNs individuais e na variabilidade da força muscular [4]. Além destas aplicações científicas importantes, modelos computacionais do sistema nervoso e muscular podem ser utilizados como ferramentas didáticas em cursos de Neurociências e Controle Motor [5].

O desenvolvimento de um modelo computacional envolve decisões relacionadas a dois aspectos principais: 1) o nível de complexidade do sistema físico ou biológico que o modelo matemático pretende representar e 2) a plataforma computacional na qual o modelo é implementado. O primeiro está associado, em algum grau, à plausibilidade biológica do modelo e à sua eficiência computacional, enquanto o segundo está relacionado ao uso de linguagens de programação livres, versáteis, de fácil aprendizado e utilização para os pesquisadores das diferentes áreas do conhecimento. Do ponto de vista do sistema neuromuscular, alguns dos modelos citados anteriormente [1-5], embora apresentem uma boa fidelidade biológica, foram desenvolvidos em plataformas proprietárias (e.g. Matlab) ou em linguagens que não são tipicamente utilizadas atualmente por pesquisadores na área de Neurociências, criando, portanto, barreiras adicionais à ampla utilização destes modelos.

Neste estudo, o objetivo é o desenvolvimento de um modelo multiescala simplificado do sistema neuromuscular que seja reproduzível. O modelo foi concebido de forma a possuir plausibilidade biológica e eficiência computacional e para a implementação computacional dos modelos foram adotadas plataformas flexíveis e reconhecidas pela comunidade científica da área de Neurociências. Além disso, os códigos-fonte e uma versão interativa estão disponíveis em https://github.com/heitorsf/reproducible.
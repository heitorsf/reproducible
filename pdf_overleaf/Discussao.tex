Neste trabalho, foi desenvolvido um modelo multiescala do sistema neuromuscular humano contendo modelos de MNs e UMs biologicamente plausíveis e computacionalmente eficientes. Para a implementação do modelo foram utilizadas a linguagem de programação Python e o simulador NEURON \cite{carnevale2006neuron}. Estas plataformas computacionais são flexíveis e reconhecidas pela comunidade científica da área de Neurociência, o que possibilitará uma ampla utilização deste modelo em estudos sobre o controle neurofisiológico da força muscular.

Pelas simulações realizadas neste estudo, pôde-se perceber que o modelo apresenta as propriedades básicas do sistema neuromuscular. Por exemplo, embora a saturação da força de uma dada UM tenha sido implementada de maneira relativamente simples, as relações força-frequência e as relações twitch/tétano são similares aos dados reportados na literatura experimental \cite{Kernell2006}.

Outros aspectos representados de forma satisfatória pelo modelo são os mecanismos de controle da força muscular (Figura 3). O fato de os modelos de MNs terem uma representação biologicamente plausível e terem sido parametrizados de acordo com o princípio do tamanho \cite{HENNEMAN1957} faz com que o recrutamento das unidades motoras e a modulação da frequência de disparos de PAs sejam compatíveis com as observações experimentais durante o controle da força isométrica em músculos esqueléticos. Além disso, a ordem de recrutamento e modulação em frequência, em que a primeira unidade motora a ser recrutada atinge frequências mais elevadas e é a última a ser derrecrutada, produz um fenômeno que é conhecido na literatura como onion skin, que parece ser um mecanismo intrínseco ao sistema neuromuscular para o controle da força \cite{DeLuca2015}.
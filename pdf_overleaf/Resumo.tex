Neste trabalho é apresentado um modelo computacional de um núcleo motor (composto por três motoneurônios alfa) e as respectivas unidades musculares. O modelo neuromuscular é constituído por unidades motoras dos tipos S, FR e FF, que foram implementadas computacionalmente utilizando a linguagem de programação Python e as bibliotecas do simulador NEURON. O modelo representou aspectos fundamentais do funcionamento do sistema neuromuscular como, por exemplo, a relação entre a força e a frequência de ativação das unidades motoras, bem como a ordem de recrutamento e a modulação da frequência de disparos de potenciais de ação das unidades motoras durante a geração da força muscular. Os códigos-fonte estão disponíveis em https://github.com/heitorsf/reproducible.